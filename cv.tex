% (c) 2002 Matthew Boedicker <mboedick@mboedick.org> (original author) http://mboedick.org
% (c) 2003-2007 David J. Grant <davidgrant-at-gmail.com> http://www.davidgrant.ca
% (c) 2008 Nathaniel Johnston <nathaniel@nathanieljohnston.com> http://www.nathanieljohnston.com
% (l) 2012 Arun I B <arunib@smail.iitm.ac.in> http://www.ee.iitm.ac.in/~ee10s026/
%This work is licensed under the Creative Commons Attribution-Noncommercial-Share Alike 2.5 License. To view a copy of this license, visit http://creativecommons.org/licenses/by-nc-sa/2.5/ or send a letter to Creative Commons, 543 Howard Street, 5th Floor, San Francisco, California, 94105, USA.


\documentclass[letterpaper,11pt]{article}
\newlength{\outerbordwidth}
\pagestyle{empty}
\raggedbottom
\raggedright
\usepackage[svgnames]{xcolor}
\usepackage{framed}
\usepackage{times}
\usepackage{tocloft}
\usepackage{graphicx}
\usepackage{multirow}
\usepackage[utf8]{inputenc}
\usepackage{tabularx}
\title{Aparna-CV}
%% package added to template
%% \usepackage[T1]{fontenc}
%% \usepackage[utf8]{inputenc}
%% \usepackage[french]{babel}
\usepackage[francais,english]{babel}
\usepackage[T1]{fontenc}
\usepackage{wrapfig}
\usepackage{enumitem} % \usepackage{enumitem}
\usepackage{setspace}
%\usepackage{comment}
\usepackage[tracing]{versions} % to get short and long version
%-----------------------------------------------------------
%Edit these values as you see fit

\setlength{\outerbordwidth}{3pt}       % Width of border outside of title bars
\definecolor{shadecolor}{gray}{0.75}   % Outer background color of title bars (0 = black, 1 = white)
\definecolor{shadecolorB}{gray}{0.93}  % Inner background color of title bars


%-----------------------------------------------------------
%Margin setup

\setlength{\evensidemargin}{-0.25in}
\setlength{\headheight}{0in}
\setlength{\headsep}{0in}
\setlength{\oddsidemargin}{-0.25in}
\setlength{\paperheight}{11in}
\setlength{\paperwidth}{8.5in}
\setlength{\tabcolsep}{0in}
\setlength{\textheight}{10in}
\setlength{\textwidth}{7in}
\setlength{\topmargin}{-0.5in}
\setlength{\topskip}{0in}
\setlength{\voffset}{0.1in}


%-----------------------------------------------------------
%Custom commands
\newcommand{\resitem}[1]{\item[] \begin{spacing}{1} #1 \end{spacing} \vspace{0pt}}

\newcommand{\resheading}[1]{\vspace{-8pt}
  \parbox{\textwidth}{\setlength{\FrameSep}{\outerbordwidth}
    \begin{shaded}
\setlength{\fboxsep}{0pt}\framebox[\textwidth][l]{\setlength{\fboxsep}{4pt}\fcolorbox{shadecolorB}{shadecolorB}{\textbf{\sffamily{\mbox{~}\makebox[6.762in][l]{\large #1} \vphantom{p\^{E}}}}}}
    \end{shaded}
  }\vspace{-17.5pt}
}

\newcommand{\ressubheading}[4]{
\begin{tabular*}{6.5in}{l@{\cftdotfill{\cftsecdotsep}\extracolsep{\fill}}r}
  \textbf{#1} & #2 \\
  \textit{#3} & \textit{#4} \\
\end{tabular*}\vspace{0pt}}


%-----------------------------------------------------------
\newcommand\picturespath{ressources/pictures/}
\newcommand\sizetopsepforbigenum{10pt}

% Sitch between English and French, thank to Remi
\newcount\IndicAnglais
\global\IndicAnglais=1
\newcommand*{\vf}{\selectlanguage{francais}\IndicAnglais=0\global\IndicAnglais=0}
\newcommand*{\vo}{\selectlanguage{english}\IndicAnglais=1\global\IndicAnglais=1}
\newcommand{\vof}[2]{\ifnum\IndicAnglais=1{}#1\else#2\fi}

%\vf % \vo = anglais / \vf = français
%\vf
\vo

\excludeversion{VERSIONlong}
%\includeversion{VERSIONlong}
%\markversion{VERSIONlong}

\begin{document}

%-----------------------------------------------------------
%Insert IIT Madras Logo 
\begin{wrapfigure}{R}{0.20\textwidth}
  \vspace{-65pt}
  \begin{center}
    \includegraphics[width=0.18\textwidth]{\picturespath photo.png}
  \end{center}
  \vspace{-10pt}
  \vspace{-100pt}
\end{wrapfigure}

\begin{tabular*}{7in}{l c r}
  \textbf{\Large Lux Benjamin}  \\
  26 ans - Permis B          & \LARGE{\vof{IT Developer}{Ingénieur en informatique}}  \\
  benjamin0lux@gmail.fr      & Dans le domaine du HPC\\ 
  06\,12\,89\,07\,20       %%&  à partir d'octobre 2013\\

  %6 rue de Candale        &  à partir d'octobre 2013\\
  %33000 Bordeaux \\
\end{tabular*}


%%%%%%%%%%%%%%%%%%%%%%%%%%%%%%
\resheading{\vof{Skills}{Compétences}}
%%%%%%%%%%%%%%%%%%%%%%%%%%%%%%
% débutant, intermédiaire, confirmé spécialiste et expert
% Novice, Beginner, Intermediate, Skillful, Proficient, Experienced, Advanced, Senior, Expert
\begin{itemize}[noitemsep,topsep=\sizetopsepforbigenum,parsep=0pt,partopsep=0pt]
\item[]\textbf{\vof{Programming Languages and parallel API:}{Langage de programmation et API pour le parallélisme:}}
  \begin{itemize}[noitemsep,topsep=0pt,parsep=0pt,partopsep=0pt]
    \resitem{{\bf \vof{Advanced:}{Confirmé :}} C, C++, Bash, MPI}
    \resitem{{\bf \vof{Intermediate:}{Intermédiaire :}} C\#, Python, Java, threads posix, SQL, \LaTeX}

    \begin{VERSIONlong}
    \resitem{{\bf \vof{Beginner:}{Notions :}}  Fortran, OpenMP, OpenCL}
    %  \item[]{Système:} Programmation système en C, multi-threads, multi-c\oe{}ur et GPU.
    \end{VERSIONlong}
  \end{itemize}
  
\item[]\textbf{Connaissances théoriques :}
  \begin{itemize}[noitemsep,topsep=0pt,parsep=0pt,partopsep=0pt]
  \item[]{Algorithmique parallèle, calcul matriciel, algorithmique des graphes, ordonnancement.}
    
    %%   \item[]{Notions de compilation, mises en application avec Yacc et
    %%     Bison.  Notions de recherche opérationnelle, mises en application
    %%     avec Cplex.  Notions de base de données avancées, mises en
    %%     applicaton avec datalog et prolog.  ,
    %%     flots et combinatoire.  Notions de gestion de projet avec le
    %%     langage UML.}
  \end{itemize}

\item[]\textbf{Environnement de travail:}
  %\item[]{Environnement de développement:}
  \begin{itemize}[noitemsep,topsep=0pt,parsep=0pt,partopsep=0pt]
  \item[]{Environnement de développement:}  Eclipse, Emacs, Visual Studio 2010/2008, Codeblocks.
  \item[]{Système d'exploitation:} Linux, Windows.% (Seven, xp).}
  \end{itemize}


%% So, on a resume, it would be completely reasonable to have a section for Language Skills that looked like this:
%%     English: native language
%%     French: limited working proficiency (ILR scale)
%%     German: full professional proficiency (ILR scale)
%% But if you think all of that is overkill, it's still ok to do this:
%%     English: native language
%%     French: intermediate (speaking, reading); basic (writing)
%%     German: fluent (speaking, reading, writing)
%%  professional and technical
\item[]\textbf{\vof{Languages:}{Langues :}}
  \begin{itemize}[noitemsep,topsep=0pt,parsep=0pt,partopsep=0pt]
  \item[]{\vof{French : native language}{Français : langue maternelle}} %  maîtrise du vocabulaire informatique (830 au Toeic)
  \item[]{\vof{English : technical reading and writing, intermediate speaking}{Anglais : lecture et écriture** technique, parlé intermédiaire}} %  maîtrise du vocabulaire informatique (830 au Toeic)
  \item[]{\vof{Spanish : experienced reading and speaking, intermediate writing}{Espagnol : lecture et parlé couramment, écrit intermédiaire}}
  \end{itemize}
\end{itemize}


%%%%%%%%%%%%%%%%%%%%%%%%%%%%%%
\resheading{\vof{Work Experience}{Expériences}}
%%%%%%%%%%%%%%%%%%%%%%%%%%%%%%

\begin{itemize}[noitemsep,topsep=\sizetopsepforbigenum,parsep=0pt,partopsep=0pt]
\item[]
  \ressubheading{Inria}{Pau}{Ingénieur Jeune Diplômé}{Novembre 2014 - 2016}
  \begin{itemize}[noitemsep,topsep=0pt,parsep=0pt,partopsep=0pt]
    \resitem{Développement de la bibliothèque Aerosol au sein de l'équipe CAGIRE. Implémentation de pré-conditionneurs et de méthodes multi-gilles par agrégation sur des maillages hybrides non-structurés.}
  \end{itemize}
\item[]
  \ressubheading{Inria}{Bordeaux}{Stage de recherche}{Mars 2013 - Septembre 2013}
  \begin{itemize}[noitemsep,topsep=0pt,parsep=0pt,partopsep=0pt]
    \resitem{Optimisation de l'ordonnancement et du partitionnement d'un couplage de code de simulation de diffusion de la chaleur en parallèle (centaines de c\oe urs), en partenariat avec le CERFACS.}
    %\resitem{Optimisation de l'ordonnancement et du partitionnement du couplage des codes hautement parallèles (12000 coeurs) de simulation de diffusion de la chaleur AVTP-AVBP (CERFACS).}
  \end{itemize}
\item[]
  \ressubheading{MaxSea International}{Bidart}{Stage de développement}{Juin 2012 - Septembre 2012}
  \begin{itemize}[noitemsep,topsep=0pt,parsep=0pt,partopsep=0pt]
    \resitem{Exportation et parallélisation d'une chaine de production de cartes géographiques marines au format vectoriel dans le Cloud Microsoft Azure.}
  \end{itemize}  
\item[] 
  \ressubheading{Expérience scolaire}{Bordeaux}{Réalisation d'une interface}{2011 - 2012}
  \begin{itemize}[noitemsep,topsep=0pt,parsep=0pt,partopsep=0pt]
    \resitem{Réalisation d'une IHM et ajout de formats de sortie pour un logiciel de traduction de Braille Musical
      pour un client, enseignant-chercheur à l'Inria.}
  \end{itemize}  
%\item[]
%  \ressubheading{Travaux Saisonniers}{France, Espagne, Irlande}{Divers emplois saisonniers : \textnormal{Travaux agricoles, menuiserie, restauration.}}{étés 2006 à 2013}
\end{itemize}

%%%%%%%%%%%%%%%%%%%%%%%%%%%%%%
\resheading{\vof{Education}{Formation}}
%%%%%%%%%%%%%%%%%%%%%%%%%%%%%%
\begin{itemize}[noitemsep,topsep=\sizetopsepforbigenum,parsep=0pt,partopsep=0pt]
  %  \ressubheading{Enseirb-Matmeca}{Bordeaux}{ \textnormal{ Diplome d'ingénieur en informatique, option calcul parallèle.}}{2010 -- 2013}
%% \item[]
%%   \ressubheading{Enseirb-Matmeca}{Bordeaux}{ \textnormal{ Diplôme d'ingénieur en informatique, option PRCD}}{2010 -- 2013}\\
%%   \hspace{4ex} (Parallélisme, Régulation et Calcul Distribué). 
\item[]
  \ressubheading{Enseirb-Matmeca}{Bordeaux}{ \textnormal{ Diplôme d'ingénieur en informatique, option Parallélisme, Régulation et Calcul Distribué. }}{2010 -- 2013}
\item[]
  \ressubheading{Lycée Louis Barthou}{Pau}{ \textnormal{ Élève en classes préparatoires, en Mathématiques et Physique, option informatique.}}{2007 -- 2010}  \\
\item[]
  \ressubheading{Lycée Victor Duruy}{Mont-de-Marsan}{ \textnormal{ Baccalauréat Scientifique avec mention assez bien.}}{2004 -- 2007}  \\
\end{itemize}

%% %%%%%%%%%%%%%%%%%%%%%%%%%%%%%%
\resheading{\vof{Leisure}{Centres d'intérêt}} % Hobby
%% %%%%%%%%%%%%%%%%%%%%%%%%%%%%%%
\begin{itemize}[noitemsep,topsep=\sizetopsepforbigenum,parsep=0pt,partopsep=0pt]
\item[] \textbf{\vof{Hobbies:}{Loisirs :}} \vof{juggling, climbing, kayaking, reading, video gaming, cycling, raspberry Pi}{ jonglerie, escalade, kayak, lecture, jeu vidéo, vélo, raspberry Pi} %kayak-polo.
\item[] \textbf{\vof{Associative:}{Associatif :}} MIPS-FabLab de Pau, Membre de l'association Jonglargonne, %logiciel libre % Ancien Membre du club de jonglage et du club Bd de l'Enseirb-Matmeca.
\end{itemize}




%% Pour sélectionner des parties, j'utilise le package ``version''.

%% Pour changer de langue, j'ai fait simplement ma propre macro \vof dont voici
%% le code :

%% \newcount\IndicAnglais
%% \global\IndicAnglais=1
%% \newcommand*{\vf}{\selectlanguage{francais}\IndicAnglais=0\global\IndicAnglais=0}
%% \newcommand*{\vo}{\selectlanguage{english}\IndicAnglais=1\global\IndicAnglais=1}
%% \newcommand{\vof}[2]{\ifnum\IndicAnglais=1{}#1\else#2\fi}

%% Tu noteras que pour respecter la typographie de chaque langue (les espaces avant les ``:'' en français
%% et pas en anglais par exemple), j'utilise ``\selectlanguage'' qui nécessite de charger
%% \usepackage[francais,english]{babel}

%% Pour avoir le texte dans les 2 langues, il suffit de taper :
%% \vof{What an interesting lecture!}{Quel cours intéressant !}

%% Pour choisir une langue, il suffit d'utiliser \vo pour l'anglais et \vf pour le français (on
%% peut passer de l'un à l'autre n'importe où dans le document).


%% Rémi 




%% %%%%%%%%%%%%%%%%%%%%%%%%%%%%%%
%% \resheading{Education}
%% %%%%%%%%%%%%%%%%%%%%%%%%%%%%%%
%% \begin{itemize}
%% \item
%%   \ressubheading{My University}{My Town, ON}{B.Sc. Physics}{2004 - 2008}
%%   \begin{itemize}
%%     \resitem{Undergraduate Thesis: Why Electron Spins Rule}
%%     \resitem{Graduated with Honours and a XX.X\% average}
%%     \end{itemize}

%% \item
%%   \ressubheading{My High School}{Hick Town, ON}{High School Diploma}{2000 - 2004}
%%   \begin{itemize}
%%     \resitem{President of Students' Council and captain of the rugby team in senior year}
%%     \resitem{Graduated with a XX.X\% average}
%%     \end{itemize}
%% \end{itemize}


%% %%%%%%%%%%%%%%%%%%%%%%%%%%%%%%
%% \resheading{Personal Statement}
%% %%%%%%%%%%%%%%%%%%%%%%%%%%%%%%
%%   \begin{center}
%%   \parbox{6.762in}{I'm the hardest working person I know, and therefore you should hire me. My qualities include blah, blah, and blah. I have lots of experience working with blah, and I am not afraid to blah. In fact, I'm such an amazing person that I can't believe you haven't hired me yet.}
%%   \end{center}


%% %%%%%%%%%%%%%%%%%%%%%%%%%%%%%%
%% \resheading{Work Experience}
%% %%%%%%%%%%%%%%%%%%%%%%%%%%%%%%
%% \begin{itemize}
%% \item
%%   \ressubheading{Calculus I}{My University}{Undergraduate Teaching Assistant}{Sept. 2007 - Apr. 2008}
%%   \begin{itemize}
%%     \resitem{Graded assignments.}
%%     \resitem{Led weekly course seminars.}
%%     \end{itemize}

%% \item 
%%   \ressubheading{Happy Time Farms}{Hick Town, ON}{Farm Hand}{Sept. 2000 - Dec. 2002}
%%   \begin{itemize}
%%     \resitem{Milked cows.}
%%     \resitem{Milked goats.}
%%     \resitem{Milked cats.}
%%     \end{itemize}

%% \item
%%   \ressubheading{Hick Town Press}{Hick Town, ON}{Paperboy}{June 1996 - Aug. 2000}
%%   \begin{itemize}
%%     \resitem{Delivered the local newspaper to 70 customers every morning.}
%%     \end{itemize}
%% \end{itemize}


%% %%%%%%%%%%%%%%%%%%%%%%%%%%%%%%
%% \resheading{Awards, Grants \& Honours}
%% %%%%%%%%%%%%%%%%%%%%%%%%%%%%%%
%% \vspace{-2pt}
%% \begin{center}\begin{tabular*}{6.6in}{l@{\extracolsep{\fill}}r}
%%     \multicolumn{2}{c}{Physics Graduation Prize (\$XXX) \cftdotfill{\cftdotsep}2007}\\
%%     \multicolumn{2}{c}{Award of Awesomeness (\$XXX) \cftdotfill{\cftdotsep} 2006}\\
%%     \multicolumn{2}{c}{My University Entrance Scholarship (\$X XXX) \cftdotfill{\cftdotsep}2004}\\
%%     \multicolumn{2}{c}{My High School Grade 12 Physics Award (\$XXX) \cftdotfill{\cftdotsep}2004}\\
%%     \multicolumn{2}{c}{Boy Genius Award (\$XXX) \cftdotfill{\cftdotsep}2003}\\
%%     \vphantom{E}
%% \end{tabular*}
%% \end{center}\vspace*{-16pt}


%% %%%%%%%%%%%%%%%%%%%%%%%%%%%%%%
%% \resheading{Skills}
%% %%%%%%%%%%%%%%%%%%%%%%%%%%%%%%
%% \begin{itemize}
%% \item
%%   Programming and Markup Languages
%%   \begin{itemize}
%%     \resitem{{\bf Expert:} CSS, HTML}
%%     \resitem{{\bf Intermediate:} Java, \LaTeX, Python}
%%     \end{itemize}

%% \item
%%   Software
%%   \begin{itemize}
%%     \resitem{{\bf Expert:} MS Access, MS Excel, MS Word}
%%     \resitem{{\bf Intermediate:} Maple, MATLAB}
%%     \end{itemize}


%% \item
%%   Other
%%   \begin{itemize}
%%     \resitem{People person, enjoys being challenged, hardworking, smells nice}
%%     \end{itemize}
%% \end{itemize}

\end{document}

%%%%%%%%%%%%%%%%%%%%%%%%%%%%%%%%%%%%%%%%%%%%%%%%%%
%% TODO : 
%%
%% reduce item space :  http://tex.stackexchange.com/questions/10684/vertical-space-in-lists
%%
%%
%%%%%%%%%%%%%%%%%%%%%%%%%%%%%%%%%%%%%%%%%%%%%%%%%%
